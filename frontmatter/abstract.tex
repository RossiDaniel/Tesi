%!TEX root = ../dissertation.tex
% the abstract

Il documento illustra il lavoro svolto durante lo stage curriculare, della durata di trecentoventi ore, presso l’azienda di trasporti Trans-Cel.\\
Lo scopo di tale stage è stato quello di realizzare dei modelli di programmazione lineare per la risoluzione dello Strip Packing Problem, questi sono poi stati utilizzati per valutare oggettivamente la bontà delle soluzioni fornite dall'euristica aziendale, la quale si occupa della risoluzione di tale problema.
Si è operato un confronto con le rispettive soluzioni ottenute con modelli matematici, messi in condizioni simili tali da operare con gli stessi vincoli imposti all'euristica.\\
Lo scopo era di fornire all'azienda uno strumento che permettesse di misurare oggettivamente la distanza delle soluzioni fornite dall'euristica dall'ottimo.\\
\newline
Gli obiettivi di maggiore importanza sono stati l'apprendimento del linguaggio di programmazione Python e delle librerie correlate, utili per la rappresentazione delle soluzioni e la manipolazione dei dati e della libreria Google Or-Tools per la prototipazione dei modelli.\\
L'obiettivo finale era la realizzazione di modelli di programmazione lineare che permettessero di produrre soluzioni corrette e ottime da confrontare con quelle fornite dall'euristica, per valutarne quantitativamente la bontà.