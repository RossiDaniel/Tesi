%!TEX root = ../dissertation.tex
% the abstract

Il documento qui di seguito illustra il lavoro svolto durante questo stage, della durata di trecentoventi ore, presso l’azienda di trasporti Trans-Cel.\\
Lo scopo di tale stage è stato di valutare oggettivamente la bontà delle soluzioni fornite dall'euristica aziendale, questo attraverso un confronto con le corrispettive soluzioni fornite dai modelli matematici, messi in condizioni tali da operare con gli stessi vincoli imposti all'euristica.\\
Lo scopo era di fornire all'azienda uno strumento che permettesse di misurare oggettivamente la distanza delle soluzioni fornite dall'euristica dall'ottimo.\\
\newline
Gli obiettivi di maggiore importanza sono stati l'apprendimento del linguaggio di programmazione \bit{Python}{python} e delle librerie correlate utili per la rappresentazione delle soluzioni e la manipolazione dei dati. La libreria \bit{Google Or-Tools}{ortools} per la prototipazione dei modelli.\\
L'obiettivo finale era la realizzazione di modelli di programmazione lineare che permettessero di produrre soluzioni reali e ottime da confrontare con quelle fornite dall'euristica, per valutarne quantitativamente la bontà.