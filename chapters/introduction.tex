%!TEX root = ../dissertation.tex
\begin{savequote}[75mm]
    Il mio motto è condivisione, sincronizzazione e azione.
    \qauthor{Filipppo Sottovia}
\end{savequote}

\chapter*{Introduzione}
\label{introduzione}
\section{L'azienda}
Trans-Cel è un'azieda di trasporti che opera nel settore da oltre trent'anni, ha una numerosa flotta composta da bilici e motrici con cui trasporta merci nel nord e centro Italia, tra le qualità che contraddistiguono questa azienda c'è la tecnica del groupage ed il trasporto di merci pericolose.\\
L'azienda ha sede ad Albignasego in provincia di Padova, qui nell'ufficio operativo vengono organizzati in tempo reale i viaggi dei mezzi per traportare le merci dei clienti, fornendo magari anche un servizio di deposito.\\
Da questa realtà si evince come si debba essere sempre pronti a rispondere tempestivamente ad ogni cliente, fornendo loro una soluzione di trasporto adeguata che soddisfi i clienti ma che permetta di far fatturare l'azienda, è da qui che nasce la necessità di un software decisionale di supporto.

\section{L'idea}
Ad oggi l'azienda ha sviluppato un'euristica che permette di organizzare al meglio le merci all'interno del container del camion.
L'idea è quella di valutare la bontà delle soluzioni fornite dall'euristica, confrontando suddette soluzioni con quelle fornite dal modello e individuare se e con quali tipi di oggetti queste riportino differenze maggiori.

\section{Organizzazione del testo}
Di seguito viene riportata per ogni capitolo una piccola descrizione delle tematiche riportate:
\begin{itemize}
\item Nel \hyperlink{(chap:processi_metodologie)}{capitolo 2} viene riportato il piano di lavoro precedentemente concordato con l'azienda.
\item Nel \hyperlink{(chap:processi_metodologie)}{capitolo 3} viene illustrato in maggiore dettaglio il problema dell'azienda e viene data una  definizione dell'astrazione di suddetto problema.
\item Nel \hyperlink{(chap:processi_metodologie)}{capitolo 4} vengono riportati i modelli utilizzati e le relative descrizioni, per ciascuno di essi verrà riportato integralmente la descrizione delle variabili utilizzate e del loro significato anche attraverso l'ausilio di materiale grafico.
\item Nel \hyperlink{(chap:processi_metodologie)}{capitolo 5} vengono riportati i modelli utilizzati e le relative descrizioni, per ciascuno di essi verrà riportato integralmente la descrizione delle variabili utilizzate e del loro significato anche attraverso l'ausilio di materiale grafico.
\item Nel \hyperlink{(chap:processi_metodologie)}{capitolo 6} vengono riportati gli strumenti adottati per lo svolgimento delle attività corredati da una breve descrizione che riporti come sono stati utilizzati.
\item Nel \hyperlink{(chap:processi_metodologie)}{capitolo 7} verrà riportato come sia stato effettuato il confronto tra euristica e modello.
\item Nel \hyperlink{(chap:processi_metodologie)}{capitolo 8} verranno riportati i risultati ottenuti e verranno tirate le conclusioni di fine stage.
\end{itemize}
\section{Convenzioni tipografiche}
Il testo adotta le seguenti convenzioni tipografiche:
\begin{itemize}
    \item ogni acronimo, abbreviazione, parola ambigua o tecnica viene spiegate e chiarificata alla fine del testo presso il glossario.
    \item ogni parola di glossario alla prima apperizione verrà etichetta come segue: $parola^{[g]}$
    \item nel riportare i modelli verranno adottate alcune convenzione riportate tra le quali   .
\end{itemize}
\section{Convenzioni modelli}
Ogni qual volta si dovrà fare riferimento ad un modello vi si farà riferimento attraverso le seguenti sigle:
\begin{itemize}
    \item 2D: modello in 2 dimensioni, considera solo lunghezza e larghezza;
    \item 2DR: modello in 2 dimensioni con la rotazione;
    \item 2DRS: modello in 2 dimensioni con rotazione e sequenza di scarico;
    \item 3D: modello in 3 dimensioni con rotazione e sovrapposizione, la rotazione è rispetto la base e si considerano larghezza, lunghezza e altezza.
\end{itemize}