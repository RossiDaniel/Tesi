%!TEX root = ../dissertation.tex

\chapter{Introduzione}
\label{introduzione}
\section{L'azienda}
Trans-Cel è un'azieda di trasporti che opera nel settore da oltre trent'anni, ha una numerosa flotta composta da bilici e motrici con cui trasporta merci nel nord e centro Italia, tra le qualità che contraddistiguono questa azienda c'è la tecnica del groupage ed il trasporto di merci pericolose.\\
L'azienda ha sede ad Albignasego in provincia di Padova, qui nell'ufficio operativo vengono organizzati in tempo reale i viaggi dei mezzi per traportare le merci dei clienti, fornendo magari anche un servizio di deposito.\\
Da questa realtà si evince come si debba essere sempre pronti a rispondere tempestivamente ad ogni cliente, fornendo loro una soluzione di trasporto adeguata che soddisfi i clienti ma che permetta di far fatturare l'azienda, è da qui che nasce la necessità di un software decisionale di supporto.

\section{L'idea}
Ad oggi l'azienda ha sviluppato un'euristica che permette di organizzare al meglio le merci all'interno del container del camion.
L'idea è quella di valutare la bontà delle soluzioni fornite dall'euristica, confrontando suddette soluzioni con quelle fornite dal modello e individuare se e con quali tipi di oggetti queste riportino differenze maggiori.
\newpage
\section{Organizzazione del testo}
Di seguito viene riportata per ogni capitolo una piccola descrizione delle tematiche trattate:
\begin{itemize}
	\item \hyperlink{(chap:processi_metodologie)}{\textbf{Capitolo 1}}: in questo capitolo vengono riportati gli obiettivi generali e la pianificazione concordata con l'azienda, inoltre vengono riportate anche le metodologie e strumenti utilizzati, infine una analisi dei rischi.
	\item \hyperlink{(chap:inquadramento)}{\textbf{Capitolo 2}}: viene descritto il problema generale e come è stato risolto dall'azienda, viene illustrato lo scopo dello stage, viene data una definizione astratta del problema dell'azienda.
	\item \hyperlink{(chap:processi_metodologie)}{\textbf{Capitolo 3}}: vengono riportati i modelli utilizzati e una descrizione dell'idea di fondo, per ciascuno di essi verrà riportato integralmente la lista delle variabili utilizzate, descrivendo inoltre anche cosa modellino, riportando per ciascun modello anche le criticità riscontrate e come sono state risolte anche attraverso l'ausilio di materiale grafico.
	\item \hyperlink{(chap:processi_metodologie)}{\textbf{Capitolo 4}}: vengono riportate le modalità con cui si sono eseguiti i test illustrando come siano stati strutturati e come sia stato possibile verificare le soluzioni fornite dai modelli.
	\item \hyperlink{(chap:processi_metodologie)}{\textbf{Capitolo 5}}: verrà data una descrizione delle funzionalità sviluppate per la verifica delle soluzioni fornite dal modello e successivamente implementate nell'euristica.
	\item \hyperlink{(chap:processi_metodologie)}{\textbf{Capitolo 6}}: vengono riportati gli strumenti adottati per lo svolgimento delle attività, corredati da una breve descrizione che riporti come sono stati utilizzati.
	\item \hyperlink{(chap:processi_metodologie)}{\textbf{Capitolo 7}}: verranno riportati i risultati ottenuti descrivendo dettagliatamente le informazioni ricavate e fornendo una serie di osservazioni sulle peculiarità dei gruppi di istanze usate nei test, infine si forniranno le conclusioni dello stage.
\end{itemize}
\section{Convenzioni tipografiche}
Il testo adotta le seguenti convenzioni tipografiche:
\begin{itemize}
	\item ogni acronimo, abbreviazione, parola ambigua o tecnica viene spiegate e chiarificata alla fine del testo presso il glossario.
	\item ogni parola di glossario alla prima apperizione verrà etichetta come segue: $parola^{[g]}$
	\item nel riportare i modelli verranno adottate alcune convenzione riportate tra le quali   .
\end{itemize}
\section{Convenzioni modelli}
Ogni qual volta si dovrà fare riferimento ad un modello vi si farà riferimento attraverso le seguenti sigle:
\begin{itemize}
	\item 2D: modello in 2 dimensioni, considera solo profondità e larghezza;
	\item 2DR: modello in 2 dimensioni con la rotazione;
	\item 2DRS: modello in 2 dimensioni con rotazione e sequenza di scarico;
	\item 3D: modello in 3 dimensioni con rotazione e sovrapposizione, la rotazione è rispetto la base e si considerano larghezza, profondità e altezza.
\end{itemize}