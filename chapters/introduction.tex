%!TEX root = ../dissertation.tex

\chapter{Introduzione}
\section{L'azienda}
Trans-Cel è un'azienda di trasporti che opera nel settore da oltre trent'anni, ha una numerosa flotta composta da bilici e motrici con cui trasporta merci nel nord e centro Italia.
Tra le qualità che contraddistinguono questa azienda c'è la tecnica del groupage ed il trasporto di merci pericolose.\\
L'azienda ha sede ad Albignasego, in provincia di Padova. Nell'ufficio operativo vengono organizzati in tempo reale i viaggi dei mezzi per trasportare le merci dei clienti, fornendo anche un servizio di deposito se richiesto.\\
Da questa realtà si evince come si debba essere sempre pronti alla comunicazione con i clienti, fornendo loro un servizio di trasporto adeguato che li soddisfi ma che permetta all'azienda di realizzare un profitto, è da qui che nasce la necessità di un software decisionale di supporto.
\newpage
\section{L'idea}
All'interno del prodotto software che sta realizzando l'azienda, è in via di sviluppo un'euristica che permette di organizzare al meglio le merci all'interno del container del camion in modo automatico. L'idea è quella di valutare la bontà delle \glo{soluzioni} fornite dalla stessa, confrontandole con quelle fornite dal modello e individuare se e con quali tipi di pacchi queste riportino differenze maggiori.

\section{Organizzazione del testo}
Di seguito viene riportata per ogni capitolo una piccola descrizione delle tematiche trattate:
\begin{itemize}
	\item \hyperlink{(chap:capitolo2)}{\textbf{Capitolo 2}}: in questo capitolo vengono riportati gli obiettivi generali e la pianificazione concordata con l'azienda, inoltre vengono riportate le metodologie e strumenti utilizzati in generale, infine un'analisi dei rischi;
	\item \hyperlink{(chap:capitolo3)}{\textbf{Capitolo 3}}: viene descritto il problema generale e come è stato risolto dall'azienda, viene illustrato lo scopo dello stage, viene data una definizione astratta del problema in esame;
	\item \hyperlink{(chap:capitolo4)}{\textbf{Capitolo 4}}: vengono riportati i modelli utilizzati e una descrizione dell'idea di base, per ciascuno di essi verrà riportato integralmente la lista delle variabili utilizzate, descrivendo inoltre anche cosa modellino, viene riportato per ciascun modello una descrizione dei vincoli, a corredo di tutto questo ci sarà del materiale grafico utilizzato durante lo stage;
	\item \hyperlink{(chap:capitolo5)}{\textbf{Capitolo 5}}: vengono riportate le modalità con cui si sono eseguiti i test, illustrando come siano stati strutturati e come sia stato possibile verificare le soluzioni fornite dai modelli;
	\item \hyperlink{(chap:capitolo6)}{\textbf{Capitolo 6}}: vengono riportati gli strumenti adottati per lo svolgimento delle attività, corredati da una breve descrizione che riporti come sono stati utilizzati;
	\item \hyperlink{(chap:capitolo7)}{\textbf{Capitolo 7}}: vengono riportati i risultati ottenuti per ciascun modello, fornendo una serie di osservazioni sulle peculiarità dei gruppi di istanze usate nei test;
	\item \hyperlink{(chap:capitolo8)}{\textbf{Capitolo 8}}: vengono riportate le conclusioni relative al numero di obiettivi soddisfatti e all'effettiva suddivisione delle ore rispetto tali obiettivi.

\end{itemize}
\section{Convenzioni tipografiche}
Il testo adotta le seguenti convenzioni tipografiche:
\begin{itemize}
	\item ogni acronimo, abbreviazione, parola ambigua o tecnica viene spiegata e chiarificata alla fine del testo;
	\item ogni parola di glossario alla prima apparizione verrà etichetta come segue: $parola^{[g]}$;
	\item ogni riga di un elenco puntato terminerà con un ; a parte l'ultima riga che si concluderà con un punto.
\end{itemize}
\section{Convenzioni modelli}
Ogni qual volta si dovrà fare riferimento ad un modello lo si farà attraverso i seguenti acronimi:
\begin{itemize}
	\item \textbf{2D}: modello in 2 dimensioni, considera solo profondità e larghezza;
	\item \textbf{2DR}: modello in 2 dimensioni con la rotazione;
	\item \textbf{2DRS}: modello in 2 dimensioni con rotazione e sequenza di scarico;
	\item \textbf{3D}: modello in 3 dimensioni con rotazione e sovrapposizione, la rotazione è la stessa disponibile nei modelli 2D, vengono considerate quindi larghezza, profondità e altezza.
\end{itemize}