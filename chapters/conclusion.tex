%!TEX root = ../dissertation.tex

\hypertarget{(chap:capitolo8)}{}
\chapter{Conclusione}
Nel seguente capitolo verranno riportate le conclusioni che si è potuto trarre alla fine di questo stage.
\section{Consuntivo finale}
Prima di iniziare lo stage era stato concordato insieme con l'azienda la pianificazione dello stesso, definendo obiettivi da raggiungere e calendarizzandoli sulle 320 ore disponibili al fine di poterli soddisfare appieno. La pianificazione iniziale è stata rispettata nella sua visione generale andando a modificarne invece le scadenze orarie, la formazione è stata più breve del previsto visto che la scelta del software di modellazione algebrica è ricaduto sullo stesso Or-Tools, per la realizzazione dei modelli in generale la prototipazione e la scrittura del modello in Or-Tools è risultata meno dispensiosa del previsto, questo ha permesso di realizzare un modello non preventivato su consiglio dell'azienda e del tutor esterno.
Si è deciso di eseguire i test di confronto con l'euristica nella parte finale dello stage invece che durante la loro realizzazione, questo perché l'intero sistema usato per testare i modelli fosse unico, concentrandosi di più nell'esecuzione di test di validazione che permettessero di verificare le soluzioni fornite dai modelli.

\begin{center}
	\begin{tabular}{|l|l|c l|}
		\hline
		\multicolumn{2}{|l|}{\textbf{Durata in ore}}		&	\multicolumn{2}{l|}{\textbf{Descrizione dell'attività}}\\
		\hline
		\multicolumn{2}{|l|}{32}	&	\multicolumn{2}{l|}{\textbf{A}: Formazione}  \\
		\hline
		\multirow{5}{1cm}{ } & 8  & \hspace{5mm}•\hspace{2mm} & Ricerca \textit{framework} di modellazione algebrica \\
		\multirow{5}{1cm}{ } & 24 & \hspace{5mm}•\hspace{2mm} & Studio Google OR - Tools                             \\
		\hline
																																					
		\multicolumn{2}{|l|}{64}	&	\multicolumn{2}{l|}{\textbf{B}: Versione algoritmo 2D}   \\
		\hline
		\multirow{5}{1cm}{ } & 8  & \hspace{5mm}•\hspace{2mm} & Individuazione ed analisi \textit{constraints}       \\
		\multirow{3}{1cm}{ } & 8  & \hspace{5mm}•\hspace{2mm} & Prototipazione modello                               \\
		\multirow{5}{1cm}{ } & 24 & \hspace{5mm}•\hspace{2mm} & Traduzione in Python - Google OR - Tools             \\
		\multirow{5}{1cm}{ } & 24 & \hspace{5mm}•\hspace{2mm} & Test validazione                                     \\	
		\hline
																																					
		\multicolumn{2}{|l|}{64}	&	\multicolumn{2}{l|}{\textbf{C}: Versione algoritmo 2DR}  \\
		\hline
		\multirow{5}{1cm}{ } & 8  & \hspace{5mm}•\hspace{2mm} & Individuazione ed analisi \textit{constraints}       \\
		\multirow{3}{1cm}{ } & 8  & \hspace{5mm}•\hspace{2mm} & Prototipazione modello                               \\
		\multirow{5}{1cm}{ } & 24 & \hspace{5mm}•\hspace{2mm} & Traduzione in Python - Google OR - Tools             \\
		\multirow{5}{1cm}{ } & 24 & \hspace{5mm}•\hspace{2mm} & Test validazione                                     \\	
		\hline
																		
		\multicolumn{2}{|l|}{72}	&	\multicolumn{2}{l|}{\textbf{C}: Versione algoritmo 2DRS}  \\
		\hline
		\multirow{5}{1cm}{ } & 8  & \hspace{5mm}•\hspace{2mm} & Individuazione ed analisi \textit{constraints}       \\
		\multirow{3}{1cm}{ } & 16 & \hspace{5mm}•\hspace{2mm} & Prototipazione modello                               \\
		\multirow{5}{1cm}{ } & 16 & \hspace{5mm}•\hspace{2mm} & Traduzione in Python - Google OR - Tools             \\
		\multirow{5}{1cm}{ } & 32 & \hspace{5mm}•\hspace{2mm} & Test validazione                                     \\	
		\hline
																																					
		\multicolumn{2}{|l|}{72}	&	\multicolumn{2}{l|}{\textbf{D}: Versione algoritmo 3D}   \\
		\hline
		\multirow{5}{1cm}{ } & 8  & \hspace{5mm}•\hspace{2mm} & Individuazione ed analisi \textit{constraints}       \\
		\multirow{3}{1cm}{ } & 16 & \hspace{5mm}•\hspace{2mm} & Prototipazione modello                               \\
		\multirow{5}{1cm}{ } & 16 & \hspace{5mm}•\hspace{2mm} & Traduzione in Python - Google OR - Tools             \\
		\multirow{5}{1cm}{ } & 32 & \hspace{5mm}•\hspace{2mm} & Test validazione                                     \\
		\hline
																																							
		\multicolumn{2}{|l|}{16}	&	\multicolumn{2}{l|}{\textbf{E}: Test confronto modello ed euristica}  \\
		\hline
		\multirow{5}{1cm}{ } & 4  & \hspace{5mm}•\hspace{2mm} & Confronto modello 2D-euristica                       \\
		\multirow{5}{1cm}{ } & 4  & \hspace{5mm}•\hspace{2mm} & Confronto modello 2DR-euristica                      \\
		\multirow{5}{1cm}{ } & 4  & \hspace{5mm}•\hspace{2mm} & Confronto modello 2DRS-euristica                     \\
		\multirow{5}{1cm}{ } & 4  & \hspace{5mm}•\hspace{2mm} & Confronto modello 3D-euristica                       \\
				
		\hline
		\multicolumn{2}{|l|}{\textbf{Totale: 320}}		&	\multicolumn{2}{l|}{}\\
		\hline
																																						
	\end{tabular}
	\captionof{table}{Ripartizione reale delle ore di stage}
\end{center}

\section{Raggiungimento degli obiettivi}
Gli obiettivi decisi prima dell'inizio dello stage e riportati nel capitolo 2, questi prevedevano il soddisfacimento di 8 obiettivi obbligatori, 4 obiettivi desiderabili e 1 opzionale, sono stati tutti soddisfatti appieno e anzi è stato realizzato un modello non preventivato.\\
L'obiettivo opzionale qui di seguito:
\begin{center}
	\textit{op01}: evoluzione euristica, fornita dall'azienda, con nuove funzionalità
\end{center}
Non è stato inserito nella pianificazione in quanto si è svolto nel mentre che si realizzavano i modelli, le nuove funzionalità sono le funzioni di verifica con cui validare le soluzioni, riportate nel capitolo 5.

\section{Conoscenze acquisite}
Le conoscenze acquiste durante il corso dello stage sono state le seguenti:
\begin{itemize}
	\item Linguaggio Python;
	\item Framework Or-Tools;
	\item Jupyter notebook;
	\item Librerie Matplotlib;
	\item Librerie Pandas;
	\item Modulo Multiprocessing;
	\item Package management attraverso pip;
\end{itemize}

\section{Valutazione personale}
La valutazione dello stage svolto presso Trans-Cel non può che dirsi positiva, ho trovato un ambiente accogliente e dei colleghi che sono poi diventati amici. Tecnicamente parlando affrontare un problema così complesso è stato formativo, poiché ho imparato ad usare strumenti utili ed un linguaggio molto popolare ed usato al giorno d'oggi, anche i lunghi confronti con il tutor interno sono stati utili per capire le meccaniche dell'euristica. Lo stage è stato la conclusione di un percorso di tre anni molto intensi che credo si possa dire abbia coronato gli sforzi fatti sia dal punto di vista tecnico sia umano.