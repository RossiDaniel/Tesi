%!TEX root = ../dissertation.tex
\begin{savequote}[75mm]
Nulla facilisi. In vel sem. Morbi id urna in diam dignissim feugiat. Proin molestie tortor eu velit. Aliquam erat volutpat. Nullam ultrices, diam tempus vulputate egestas, eros pede varius leo.
\qauthor{Quoteauthor Lastname}
\end{savequote}
\hypertarget{(chap:processi_metodologie)}{}
\chapter{Processi e metodologie}
In questo capitolo verranno riportati in modo approfondito lo scopo e gli obiettivi dello stage, riportando lo scadenziario delle attività e contestualizzandole alla relatà dell'azienda.

\section{Contesto}
Il progetto generale nasce dalla visione di Filippo Sottovia, titolare dell'azienda, di realizzare un software decisionale di supporto che permettesse di sopperire a molteplici obiettivi quali:
\begin{itemize}
    \item agevolazione degli operatori nello svolgimento delle loro mansioni;
    \item facilitazione del processo decisionale richiedendo così un lavoratore meno esperto;
    \item nuove attività frutto del tempo risparmiato grazie all'aumento della produttività;
    \item condivisione in tempo reale delle informazioni sullo stato dei trasporti;
    \item stima di costi e profitti disponibile in ogni momento.
\end{itemize}
\noindent Il software, fornisce un'interfaccia grafica web intuitiva che fa dell'usabilità la propria punta di diamante, essa infatti persegue l'idea di rendere il più semplice possibile operare sul sistema senza però peccare di professionalità, il sistema inoltre è equipaggiato con algoritmi che permettono di ottimizzare gli ordini organizzando al meglio le merci da trasportare ripartendole ai camion della flotta, tenendo conto degli orari di carico e scarico delle sedi dei clienti/fornitori, delle condizioni di traffico sulle strade, delle pause dovute per legge agli autisti e della particolarità delle merci.

\section{Introduzione al progetto}
L'azienda per permettere di stimare lo spazio occupato dalle merci ha sviluppato un'euristica, questa riceve in input le merci da trasportare divise nei diversi ordini, le approssima a parallelepipedi e le dispone al meglio sul pianale del camion, sovrapponendole dove possibile, con l'obiettivo di ridurre al minimo lo spazio lineare occupato dalle stesse, così facendo si risparmia spazio per eventuali altre merci da caricare qualora arrivassero nuovi ordini.
A rendere ancora più complicato il lavoro dell'euristica riportiamo tre principali problemi da tenere in considerazione:
\begin{itemize}
    \item Stabilità degli oggetti: la faccia inferiore di ciascun oggetto deve poggiare per intero sulle facce superiori di altri oggetti sotto di sè o sul pianale del camion;
    \item Sequenza di scarico: ogni oggetto deve poter essere scaricato lateralemente o frontalmente, questo implica che non può avere in torno a sè merci che ne blocchino lo scarico, in quanto appartenenti ad ordini maggiori;
    \item Baricentro degli oggetti: peso delle merci deve essere equamente distribuito sul pianale del camion.
\end{itemize}

\section{Vincoli temporali, tecnologici e metodologici}
Nel periodo di stage svolto presso l'azienda mi è stato chiesto di tenere un \textit{diario} condiviso utilizzando \textit{dropbox}, nel suddetto diario mi si richiedeva di annotare giornalieramente l'avanzare del lavoro riportando l'avanzamento dei lavori, idee, osservazione riguardanti criticità rilevate e positività riscontrate negli strumenti utilizzati, nella cartella condivisa mi è stato chiesto di includere anche gli articoli accademici letti e le presentazioni fatte in azienda man mano che si proseguiva con lo stage. L'ambiente di lavoro si è dimostrato molto stimolante ed è stato molto utile confrontarsi con i colleghi per ricevere opinioni e consigli su come indirizzare il lavoro. Ogni mattina prima di iniziare a lavorare facevamo un rapido \textit{brainstorming} individuando gli obiettivi della giornata e per fare il punto.
Un'altra cosa importante è stata la presentazione di metà stage, alla quale ha presenziato tutto il team di sviluppo, parte del personale d'amministrazione e il Professor De Giovanni che mi ha fornito preziosi consigli su come orientare gli sforzi al meglio per la seconda metà dello stage.
Prima di iniziare lo stage è stato concordato con l'azienda un piano di lavoro su un totale di 320 ore, lavorando 5 giorni a settimana, 8 ore per ciascun giorno. 

\section{Requisiti e obiettivi}
Nella tabella riportata di seguito vengono riportati gli obiettivi dello stage, a corredo degli stessi vi sarà un codice univoco ed una breve descrizione. Ogni obiettivo è provvisto di un codice che lo identifica formato da una delle seguenti stringhe \textbf{ob},de,op] rappresentante il livello di importanza e da un numero incrementale positivo, che rispetta la seguente nomenclatura: [importanza][identificativo].\\ 
Il livello di importanza di ciascun obiettivo può essere uno tra i seguenti:
\begin{itemize}
    \item Obbligatorio: individuato dalla stringa \textit{ob}, sono obiettivi fondamentali per la riuscita del progetto, il loro soddisfacimento dovrà verificarsi assolutamente entro la fine dello stage, pena il fallimento dello stesso;
    \item Desiderabile: individuato dalla stringa \textit{de}, sono obiettivi secondari su cui però si nutre dell'interesse, il loro soddisfacimento è auspicabile entro la fine dello stage;
    \item Opzionale: individuato dalla stringa \textit{op}, sono obiettivi di contorno su cui si nutre poco interesse, la loro realizzazione si verificherà nel momento in cui si dovesse soddisfare tutti gli obiettivi obbligatori e desiderabili prima della fine dello stage.
\end{itemize}

Si prevede lo svolgimento dei seguenti obiettivi:
\begin{itemize}
	\item Obbligatori
	\begin{itemize}
		\item \underline{\textit{ob01}}: individuazione e analisi \textit{constraints} modello 2D;
		\item \underline{\textit{ob02}}: modello matematico 2D con \textit{framework} di modellazione algebrica;
		\item \underline{\textit{ob03}}: traduzione modello in Python con l'ausilio di Google OR-Tools;
		\item \underline{\textit{ob04}}: test sul modello 2D e confronto con l'euristica;
		\item \underline{\textit{ob05}}: individuazione e analisi \textit{constraints} modello 2DR; 
		\item \underline{\textit{ob06}}: modello matematico 2DR con \textit{framework} di modellazione algebrica;
		\item \underline{\textit{ob07}}: traduzione modello in Python con l'ausilio di Google OR-Tools;
		\item \underline{\textit{ob08}}: test sul modello 2DR e confronto con l'euristica;
    \end{itemize}
	\item Desiderabili
	\begin{itemize}
		\item \underline{\textit{de01}}: individuazione e analisi \textit{constraints} modello 3D;
		\item \underline{\textit{de02}}: modello matematico 3D con \textit{framework} di modellazione algebrica;
		\item \underline{\textit{de03}}: traduzione modello in Python con l'ausilio di Google OR-Tools;
		\item \underline{\textit{de04}}: test sul modello 3D e confronto con l'euristica;
	\end{itemize}
	\item Opzionali
	\begin{itemize}
		\item \underline{\textit{op01}}: evoluzione euristica, fornita dall'azienda, con nuove funzionalità;
	\end{itemize} 
\end{itemize}

\section{Pianificazione}
Con le ore a disposizione per questo stage si è proceduto a organizzare come segue le attività:
\begin{itemize}
    \item Formazione: si è resa necessaria una formazione che spaziasse dalla programmazione lineare intera, al linguaggio Python, al framework Or-Tools e al solver CBC;
    \item Preparazione ambiente: periodo in cui ho realizzato dei piccoli modelli di esempio con cui impratichirmi il tutto interfacciandomi con Python;
    \item Realizzazione modelli: durante questo periodo si è provveduto a analizzare, realizzare e testare i modelli richiesti;
    \item Verifica: si è provveduto a verificare che le soluzione fornite dal modello fossero corrette;
    \item Testing: si è provveduto a utilizzare i modelli confrontandoli con l'euristica fornita dall'azienda;
    \item Documentazione risultati: si è provveduto a documentare  
\end{itemize}
Viste le ore a disposizione dello stage gli obiettivi sono stati ripartiti in tre principali macroperiodi, uno per modello, a loro volta divisi nei seguenti periodi.
