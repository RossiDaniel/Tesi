%!TEX root = ../dissertation.tex

\hypertarget{(chap:capitolo5)}{}
\chapter{Test e validazione}
Una attività di fondamentale importanza è stata quella di testing dei modelli.
\section{Istanze}
Diamo prima una definizione di istanza:
\begin{center}
	Una \textbf{istanza} è un insieme formato dai pacchi da disporre nel contenitore.
\end{center}
Diamo inoltre la definizione di gruppo di istanze:
\begin{center}
	Un \textbf{gruppo di istanze} è un insieme di istanze accomunate tra loro dalle simili dimensioni dei pacchi o dal numero di oggetti.
\end{center}
La prassi generale nell'eseguire il testing dei modelli è stata quella di creare un centinaio di istanze casuali, eseguire il modello sulle istanze generate e verificare la correttezza delle soluzioni fornite, con l'obiettivo di individuare imperfezioni e testare la tenuta dei vincoli.
La verifica della correttezza delle soluzioni veniva fatta attraverso funzioni realizzate appositamente che verranno illustrate nelle sezioni successive.
Questa procedura richiedeva circa una ventina di minuti di esecuzione ed è stata utilizzata solo dopo che i modelli avessero espresso, attraverso i loro vincoli, correttamente tutti i requisiti di partenza. La loro esecuzione è avvenuta su istanze con pochi oggetti per risparmiare tempo e individuare subito eventuali errori.

Dopo la realizzazione di tutti i modelli, sicuri delle soluzioni fornite grazie alla validazione effettuata, si è passati all'esecuzione di 100 test per ciascun gruppo di istanze, riportate di seguito per ottenere maggiori informazioni in merito, il tutto eseguito per ciascun modello.
\begin{center}
	\begin{tabular}{lrrrr}
		\toprule
		{}
		\#ist & width\_a & width\_b & depth\_a & depth\_b \\
		\midrule
		0     & 0.5      & 2.45     & 0.5      & 2.45     \\
		1     & 0.5      & 1.50     & 0.5      & 4.00     \\
		2     & 1.5      & 2.45     & 0.5      & 4.00     \\
		3     & 0.5      & 1.50     & 3.0      & 4.00     \\
		4     & 1.5      & 2.45     & 3.0      & 4.00     \\
		5     & 0.1      & 1.00     & 0.1      & 1.00     \\
		6     & 0.1      & 1.00     & 3.0      & 4.00     \\
		7     & 2.0      & 2.45     & 3.0      & 4.00     \\
		8     & 2.0      & 2.45     & 2.0      & 2.45     \\
		9     & 0.1      & 1.00     & 0.1      & 4.00     \\
		\bottomrule
	\end{tabular}
	\captionof{table}{Le dimensioni dei gruppi di istanze eseguite}
	\label{tab:istanze}
\end{center}

Nella tabella \ref{tab:istanze} l'indice \#ist è un codice identificativo del gruppo di istanze. Vengono riportati anche gli intervalli entro cui si è deciso debbano rientrare le dimensioni casuali dei pacchi per ogni gruppo di istanze, gli intervalli sono stati così definiti:
\begin{itemize}
	\item \textbf{Larghezza}: [width\_a, width\_b]
	\item \textbf{Profondità}: [depth\_a, depth\_b]
\end{itemize}

\section{Numero di oggetti}
Fin dal modello 2D abbiamo capito che il numero di pacchi per ogni istanza sarebbe dovuto essere limitato in quanto già con sette pacchi i tempi di esecuzione aumentavano vertiginosamente, abbiamo quindi deciso di scegliere in modo casuale in un intervallo [3,10] il numero di pacchi da assegnare a ciascuna istanza, imponendo un \glo{time limit} di 300 secondi.
Questo portava ad ottenere soluzioni ottime e soluzioni \glo{best bound}, l'analisi dei risultati sarà divisa in queste due categorie.

\begin{figure}[!ht]
	\begin{center} \includegraphics[scale=0.8]{figures/time_nitems}
		\caption[Grafico tempi esecuzione]{Grafico tempi esecuzione modello con diversi oggetti}  
		\label{fig:times}
	\end{center}
\end{figure}

Nel grafico in Figura \ref{fig:times} viene riportato come ascissa il numero di oggetti e come ordinata il tempo espresso in secondi, mostrando che aumentando il numero di oggetti cresca notevolmente il tempo di esecuzione, passando da pochi secondi a minuti.

Risultano molto interessanti quelle istanze che nonostante abbiano molti oggetti richiedano meno tempo per essere eseguiti, forse questo dovuto alle particolari dimensioni degli oggetti.

\section{Validazione soluzioni fornite}
Di fondamentale importanza per il confronto è stato che le soluzioni fornite dal modello fossero corrette, questo ha richiesto lo sviluppo di alcune funzioni che controllassero la validità delle stesse. 
\newpage
Per fare ciò sono state realizzate le seguenti funzioni:
\begin{itemize}
	\item \textbf{CheckFeasible}: utilizzata per controllare che non vi fossero pacchi intersecati tra loro;
	\item \textbf{CheckSequence}: realizzata in due versioni:
	      \begin{itemize}
	      	\item \textbf{2D}: considerando che tutti gli oggetti fossero alla stessa altezza controllava che ciascun pacco avesse almeno una via attraverso cui essere scaricato;
	      	\item \textbf{3D}: considerava anche la sovrapposizione tra pacchi e oltre a controllare di avere almeno una delle tre direzioni libere rispetto l'altezza a cui si trovava, controllava anche se sopra ciascun pacco al momento dello scarico vi fossero altri pacchi;
	      \end{itemize}
	\item \textbf{CheckStable}: utilizzata per controllare la stabilità degli oggetti, verificava che l'area di base inferiore poggiasse completamente sulle basi superiori degli oggetti sottostanti;
	\item \textbf{ProjectionCommonArea}: utilizzata per verificare se due parallelepipedi avessero intersezioni tra le loro proiezioni rispetto ad una coppia di assi come altezza e profondità.
\end{itemize}

Inoltre per natura stessa dei modelli essi ricercano la soluzione ottima. Si può quindi dire che se le soluzioni trovate vengono individuate entro il tempo limite e vengono verificate dalle funzioni precedentemente riportate, allora queste possono dirsi effettivamente ottime e reali, perfette candidate per un confronto con le soluzioni corrispettive fornite dall'euristica.