%!TEX root = ../dissertation.tex

\hypertarget{(chap:capitolo7)}{}
\chapter{Risultati}
Nel seguente capitolo riporteremo i risultati ottenuti dai confronti tra le soluzioni fornite dai modelli e le corrispondenti fornite dall'euristica.\\
Per ciascun modello verranno riportate due tabelle con i risultati ottenuti, queste verranno divise nelle due categorie di soluzioni ottenute:
\begin{itemize}
	\item ottime;
	\item best bound.
\end{itemize} 
Per best bound viene intesa una soluzione fornita da un modello dopo che si è raggiunto il suo time limit, ossia tempo massimo di esecuzione, questa soluzione non è ottima ma può essere considerata come un buon risultato.
Le tabelle riporteranno il numero di istanze per gruppo, errore relativo $\epsilon_r$ nel rapporto la tra i metri lineari ottenute, errore assoluto $\epsilon_a$ espresso in metri e il tempo medio di esecuzione di quel gruppo di istanze. Successivamente verranno riportate alcune osservazioni sui risultati di alcuni gruppi di istanze e un rapporto generale sulla bontà delle soluzioni dei modelli aldilà dei gruppi di istanze, ricordiamo che sia il modello che l'euristica venivano eseguite sulla stessa istanze e con le stesse caratteristiche, questo permette un confronto oggettivo.

\newpage
\section{Errori}
L'obiettivo sia del modello che dell'euristica è quello di minimizzare i metri lineari occupati dai pacchi, per convenzione vengono così deiniti:
\begin{itemize}
	\item Metri lineare modello: $Obj_m$
	\item Metri lineare euristica: $Obj_h$
\end{itemize}

Gli errori che verranno riportati successivamente vengono calcolati come segue:
\begin{center}
	$$\epsilon_a = Obj_h - Obj_m$$
	$$\epsilon_r = \frac{100 (Obj_h - Obj_m) }{Obj_m}$$
\end{center}

\newpage
\section{Risultati modello 2DR}
Di seguito verranno riportati i risultati ottenuti dal confronto tra soluzioni fornite dal modello e dall'euristica:
\begin{center}
	\begin{table}[H]
		\begin{minipage}{0.4\textwidth}
			\centering
			\begin{tabular}{lrrrr}
				\toprule
				{} & \#ist & $\epsilon_r$ & $\epsilon_a$ & Time  \\
				\midrule
				0  & 64.0  & 3.89         & 0.23         & 40.95 \\
				1  & 73.0  & 11.90        & 0.81         & 31.51 \\
				2  & 76.0  & 0.94         & 0.10         & 19.76 \\
				3  & 84.0  & 12.29        & 1.26         & 19.79 \\
				4  & 75.0  & 0.00         & 0.00         & 27.69 \\
				5  & 73.0  & 14.17        & 0.11         & 12.58 \\
				6  & 78.0  & 6.60         & 0.47         & 20.95 \\
				7  & 76.0  & 0.00         & 0.00         & 36.62 \\
				8  & 81.0  & 0.00         & 0.00         & 23.70 \\
				9  & 81.0  & 10.34        & 0.45         & 10.60 \\
				\bottomrule
			\end{tabular}
			\captionof{table}{Risultati 2DR ottimi}
		\end{minipage}
		\begin{minipage}{0.5\textwidth}
			\centering
			\begin{tabular}{lrrrr}
				\toprule
				{} & \#ist & $\epsilon_r$ & $\epsilon_a$ \\
				\midrule
				0  & 36.0  & 7.35         & 0.59         \\
				1  & 27.0  & 15.98        & 1.48         \\
				2  & 24.0  & 0.91         & 0.17         \\
				3  & 16.0  & 17.26        & 2.62         \\
				4  & 25.0  & 0.00         & 0.00         \\
				5  & 27.0  & 16.16        & 0.21         \\
				6  & 22.0  & 19.40        & 1.70         \\
				7  & 24.0  & 0.00         & 0.00         \\
				8  & 19.0  & 0.00         & 0.00         \\
				9  & 19.0  & 20.58        & 1.10         \\
				\bottomrule
			\end{tabular}
			\captionof{table}{Risultati 2DR best bound}
		\end{minipage}
	\end{table}
\end{center}
Le prime osservazioni che mi sento di condividere sono sul perché gruppi di istanze come il 4,7 e 8 riportino errori $\epsilon_r$ e $\epsilon_a$ uguali a 0, questo perché mediamente i pacchi di tali gruppi di istanze avevano dimensioni che non permettevano di essere affiancati tra loro, quindi per ogni istanza il modello e l'euristica si sono ridotti ad allineare i pacchi uno davanti all'altro, ottenendo sempre lo stesso risultato. Sembra interessante anche il gruppo di istanze 9 che riporta mediamente un tempo di esecuzione minore, questo gruppo di istanze è formata da mix di pacchi stretti e corti e stretti e lunghi, questo porta gli errori $\epsilon_r$ per quanto riguarda le soluzioni ottime ad un valore del 10\%, mentre invece con le soluzioni best bound questo aumenta fino al 20\%.
Considerazione interessante inoltre è come ci si aspetterebbe di veder diminuire gli errori con le soluzioni best bound, questo invece non accade e anzi aumenta il distacco dalle soluzioni fornite dall'euristica, segno che le soluzioni best bound fornite dal modello sono comunque valide per il confronto.

\newpage
\section{Risultati modello 2DRS}
Di seguito verranno riportati i risultati ottenuti dal confronto tra soluzioni fornite da euristica e modello 2DRS:
\begin{center}
	\begin{table}[H]
		\begin{minipage}{0.45\textwidth}
			\centering
			\begin{tabular}{lrrrr}
				\toprule
				{} & \#ist & $\epsilon_r$ & $\epsilon_a$ & Time  \\
				\midrule
				0  & 76.0  & 4.74         & 0.26         & 41.72 \\
				1  & 75.0  & 12.44        & 0.83         & 27.96 \\
				2  & 82.0  & 0.56         & 0.07         & 20.97 \\
				3  & 72.0  & 14.59        & 1.62         & 29.67 \\
				4  & 81.0  & 0.00         & 0.00         & 28.00 \\
				5  & 75.0  & 14.87        & 0.12         & 17.89 \\
				6  & 78.0  & 9.18         & 0.68         & 17.22 \\
				7  & 85.0  & 0.00         & 0.00         & 13.99 \\
				8  & 79.0  & 0.00         & 0.00         & 25.25 \\
				9  & 89.0  & 7.69         & 0.32         & 12.87 \\
				\bottomrule
			\end{tabular}
			\captionof{table}{Risultati 2DRS ottimi}
		\end{minipage}
		\begin{minipage}{0.5\textwidth}
			\centering
			\begin{tabular}{lrrr}
				\toprule
				{} & \#ist & $\epsilon_r$ & $\epsilon_a$ \\
				\midrule
				0  & 24.0  & 5.84         & 0.52         \\
				1  & 25.0  & 14.26        & 1.29         \\
				2  & 18.0  & 0.99         & 0.18         \\
				3  & 28.0  & 15.82        & 2.40         \\
				4  & 19.0  & 0.00         & 0.00         \\
				5  & 25.0  & 16.20        & 0.20         \\
				6  & 22.0  & 20.83        & 1.97         \\
				7  & 15.0  & 0.00         & 0.00         \\
				8  & 21.0  & 0.00         & 0.00         \\
				9  & 11.0  & 23.35        & 1.06         \\
				\bottomrule
			\end{tabular}
			\captionof{table}{Risultati 2DRS best bound}
		\end{minipage}
	\end{table}
\end{center}
Le prime osservazioni da riportare sono che l'euristica nel disporre i pacchi rispettando la corretta sequenza di scarico, li inserisce ordinando i pacchi in base all'ordine di scarico così da essere sicuro che ciascun pacco non abbia dietro a sé pacchi di ordini successivi, questo però preclude la possibilità di avere quelle soluzioni simili a quella riportata per esempio nella Figura \ref{fig:2drs_abg}, al contrario però porta ad avere anche soluzioni in cui non sia garantita la stabilità generale degli oggetti. Possiamo osservare come un aumento della complessità del problema non abbia portato però ad un aumento degli errori generalizzato ma anzi in alcuni gruppi di istanze sia diminuito, mentre i tempi d'esecuzione siano aumentati generalmente. Le istanze con gli errori maggiori restano sempre le stesse del precedente modello, in particolare salta all'occhio come l'errore $\epsilon_r$ del gruppo di istanze 3 si attesti al 20\%, questo gruppo è formato da pacchi stretti e lunghi, valore importante che porta ad uno scarto tra modello e euristica di quasi 3 metri.

\newpage
\section{Risultati modello 3D}
Di seguito verranno riportati i risultati ottenuti dal confronto tra soluzioni fornite da euristica e modello 3D con rotazione e sovrapposizione:
\begin{center}
	\begin{table}[H]
		\begin{minipage}{0.4\textwidth}
			\centering
			\begin{tabular}{lrrrr}
				\toprule
				{} & \#ist & $\epsilon_r$ & $\epsilon_a$ & Time  \\
				\midrule
				0  & 76.0  & 5.05         & 0.25         & 39.15 \\
				1  & 77.0  & 11.11        & 0.67         & 31.87 \\
				2  & 84.0  & 0.84         & 0.10         & 33.15 \\
				3  & 72.0  & 9.83         & 0.96         & 34.07 \\
				4  & 79.0  & 0.00         & 0.00         & 36.59 \\
				5  & 65.0  & 9.90         & 0.08         & 13.53 \\
				6  & 76.0  & 6.33         & 0.45         & 26.50 \\
				7  & 76.0  & 0.00         & 0.00         & 39.09 \\
				8  & 78.0  & 0.00         & 0.00         & 25.45 \\
				9  & 76.0  & 10.73        & 0.37         & 20.24 \\
				\bottomrule
			\end{tabular}
			\captionof{table}{Risultati 3D ottimi}
		\end{minipage}
		\begin{minipage}{0.5\textwidth}
			\centering
			\begin{tabular}{lrrr}
				\toprule
				{} & \#ist & $\epsilon_r$ & $\epsilon_a$ \\
				\midrule
				0  & 24.0  & 5.24         & 0.49         \\
				1  & 23.0  & 12.87        & 1.22         \\
				2  & 16.0  & 0.80         & 0.11         \\
				3  & 28.0  & 21.01        & 2.95         \\
				4  & 21.0  & 0.00         & 0.00         \\
				5  & 35.0  & 14.47        & 0.17         \\
				6  & 24.0  & 16.85        & 1.55         \\
				7  & 24.0  & 0.00         & 0.00         \\
				8  & 22.0  & 0.00         & 0.00         \\
				9  & 24.0  & 13.96        & 0.59         \\
				\bottomrule
			\end{tabular}
			\captionof{table}{Risultati 3D best bound}
		\end{minipage}
	\end{table}
\end{center}
Le istanze che sono state generate non contenevano solo oggetti \glo{stackable}, ma anche oggetti \textit{fragili} che sopra di sé non potevano averne nessun altro, quindi nell'ordinare gli oggetti l'euristica inseriva nel container prima gli oggetti \textit{fragili} comportando così che sopra di essi non potessero esservi altri oggetti. Il modello in questione risulta essere più vincolato dell'euristica, in quanto non è stato possibile esprimere il concetto di stabilità degli oggetti a pieno. Il concetto è stato semplificato in modo che, se un oggetto si trova al di sopra di un altro, allora la sua base inferiore deve essere completamente poggiata alla faccia superiore di quello sotto. L'euristica aveva invece la possibilità che lo stesso pacco potesse poggiare su più pacchi, nei test abbiamo tenuto conto del problema inserendo solo alcuni pacchi stackable.