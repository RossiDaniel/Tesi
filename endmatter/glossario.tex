%!TEX root = ../dissertation.tex

\begin{itemize} 
	\item \gloTarget{agile} nasce per rispondere al troppo lungo periodo di sviluppo del SW, utilizzando rilasci rapidi e incrementali, che si adatino dinamicamente indicato per dare rapidamente ai clienti camente ai nuovi requisiti. E un prodotto. Nasce per svincolarsi dall'eccessiva rigidità degli altri modelli di vita;
	\item \gloTarget{best bound} è una soluzione fornita da un modello dopo che si è raggiunto il tempo limite di esecuzione, questa soluzione non è ottima ma può essere considerata come un buon risultato;
	\item \gloTarget{brainstorming} incontri di gruppo la denizione di nuove idee. Servono partecipanti, dove il indica uno che appunti ciò che si dice e uno che passi la parola ad uno alla volta i partecipanti per evitare che i caratteri forti si impongano;
	\item \gloTarget{ciclo di vita} gli stati che il prodotto assume dal concepimento al ritiro;
	\item \gloTarget{commerciale} prodotto realizzato allo scopo di trarne un profitto o che è adatto a scopi commerciali.
	\item \gloTarget{euristica} algoritmo progettato per risolvere un problema più velocemente, spesso una strada obbligata per risolvere problemi molto difficili
	\item \gloTarget{GIL} mechanism used in computer-language interpreters to synchronize the execution of threads so that only one native thread can execute at a time
	\item \gloTarget{incremento} particolare tipo di iterazione dove il successo è garantito.
	\item \gloTarget{modello incrementale} modello di ciclo di vita che prevede rilasci multipli e successivi, dove ciascuno di essi realizza un incremento di funzionalità. I requisiti vengono trattati per importanza, prima quelli di maggior importanza in modo che possano stabilizzarsi con il rilascio delle versioni fino a quelli minori.
	\item \gloTarget{milestone} punto nel tempo al quale associamo un insieme di stati di avanzamento
	\item \gloTarget{open source} termine utilizzato per riferirsi ad un software di cui i detentori dei diritti sullo stesso ne rendono pubblico il codice sorgente,
	\item \gloTarget{project management} intende l'insieme delle attività di back office e front office aziendale, svolte tipicamente da una o più figure dedicate e specializzate dette project manager, volte all'analisi, progettazione, pianificazione e realizzazione degli obiettivi di un progetto
	\item \gloTarget{slack} tempo aggiuntivo ad un'attività che ha lo scopo di evitare ritardi nella produzione del prodotto.
	\item \gloTarget{soluzioni}  termine usato per indicare l'insieme di contenitore e pacchi da disporre con le relative posizioni rispetto il contenitore
	\item \gloTarget{solver} software commerciale e non che permette di risolvere problemi di programmazione lineare
	\item \gloTarget{stackable} termine utilizzato per indicare se un pacco piuò avere sopra di sè altri pacchi
	\item \gloTarget{time limit} termine usato per indicare un tempo limite entro il quale può essere ricercata una soluzione dal solver.
	\item \gloTarget{vehicle routing} famiglia di problemi che trattano tutti gli aspetti della gestione di una flotta di veicoli nell'ambito della logistica.
\end{itemize} 
